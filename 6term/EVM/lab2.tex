\documentclass[12pt]{article}
\usepackage[utf8x]{inputenc}
\usepackage{cmap}
\usepackage{soul}
\usepackage{ulem}
\usepackage[english,russian]{babel}
\usepackage{enumerate}
\pagestyle{empty}
	\hoffset=0mm
	\oddsidemargin=-15mm
	\topmargin=-15mm

	\headheight=0pt
	\headsep=0pt
	\textheight=28cm
	\textwidth=16cm
	\marginparsep=0pt
	\marginparwidth=0pt
	\footskip=0pt
	\marginparpush=0pt
	\hoffset=0mm
	\voffset=0mm
	\parindent=0.5cm
	\hangindent=0.5cm 
	\hangafter=2
	%{\raggedright}
%\renewcommand{\baselinestretch}{1}

\begin{document}

\textbf{Вариант 5.}
\begin{flushleft}
\framebox(550,740)[1]{

\begin{minipage}[0]{540pt}

\begin{flushright}
{\large \uline{\textbf{Кратные интегралы}}}

\end{flushright}
\par{\ }
%\vspace*{-10pt}

%\hspace{3mm}
	\parindent=0.5cm
\hangindent=-0.5cm 
	\hangafter=0


\textbf{1 Свойства кратных интегралов}


$$\hspace*{-11.5cm}\int\limits_{D}{1 \cdotp dx_1 \cdotp dx_2 \cdotp ... \cdotp dx_n = mD}$$

\vspace*{-1cm}
\textbf{1)}

$$\hspace*{-0.5cm}
\int\limits_{D}
{[A \cdotp f(x) + B \cdotp g(x)]dx} = 
A \cdotp \int\limits_{D}{f(x)dx} + 
B \cdotp \int\limits_{D}{g(x)dx}$$

\vspace*{-1cm}
\textbf{2) \uline{\textit{Линейность}}}

$$\hspace*{1.5cm} 
\int\limits_{D}{f(x)dx} \leq 
\int\limits_{D}{g(x)dx}$$

\vspace*{-1.0cm}
\textbf{3)} Если $f(x) \leq g(x)$ в области  D, тогда 

$$\hspace*{0.5cm}
-\int\limits_{D}{g(x)dx} \leq 
\int\limits_{D}{f(x)dx} \leq 
\int\limits_{D}{g(x)dx}$$

\vspace*{-1.0cm}
\textbf{4)} Если $\vert f(x) \vert \leq g(x)$, тогда 

$$\hspace*{-11.5cm} 
\left| \int\limits_{D}{f(x)dx} \right|\leq 
\int\limits_{D}{\vert f(x) \vert dx} 
$$

\vspace*{-1.0cm}
\textbf{5)}  

$$ $$

\vspace*{-0.5cm}
\textbf{6)}  \uline{\textit{Теорема о среднем}}
\par{Пусть функция $f(x) = f(x_1,x_2,...,x_n)$ непрерывна в области D, тогда $\exists \xi \in D$, такая, что}

$$\hspace*{-4.0cm} 
 \int\limits_{D}{f(x)dx} \leq f( \xi )mD,\ \  
m \leq f(x) \leq M,\ \  m = \min_{x \in D}f(x), \ \ 
M = \max_{x \in D}f(x)
$$
%\vspace*{-1.0cm}
$$\hspace*{2.7cm} 
m(mD) \leq \int\limits_{D}{f(x)dx} \leq M(mD) \Rightarrow
m \leq \int\limits_{D}{\frac{f(x)}{mD}dx} \leq M \Rightarrow \exists \xi \in D \mbox{ такая, что} $$


\vspace*{-1.3cm}
\par{По 3) свойству} 
\vspace*{0.3cm}
$$\hspace*{-14cm} 
\int\limits_{D}{\frac{f(x)}{mD}dx} \leq f( \xi )
$$

$$ $$

\vspace*{-1.0cm}
\textbf{7)}  \uline{\textit{Интеграл от непрерывной функции}} $f(x) \exists$ в области D, если $f(x)$ непрерывна в D, тогда 

$$\hspace*{-15.5cm}
\exists \int\limits_{D}{f(x)dx}
$$

$$ $$

\vspace*{-1.0cm}
\textbf{2 Двойной интеграл.} 
\uline{\textit{Сведение к повторному интегралу}}
\\
\par{Перестановка пределов интегрирования}
\par{Интеграл по прямоугольнику D}

\vspace*{-0.3cm}
\begin{equation}
\hspace*{-7.0cm}
\int\limits_{a}^{b}{\int\limits_{c}^{d}{f(x,y)dxdy}} = 
\int\limits_{c}^{d}{dy \int\limits_{a}^{b}{f(x,y)dx}} = 
\int\limits_{a}^{b}{dx \int\limits_{c}^{d}{f(x,y)dy}}  
\end{equation}

\begin{equation}
\hspace*{-10.5cm}
\int{}
\hspace*{-0.3cm}
\int\limits_{\hspace*{-0.3cm}(D)}{f(x,y)dxdy} = 
\int\limits_{a}^{b}{dx \int\limits_{v(x)}^{w(x)}{f(x,y)dy}}  
\end{equation}
\\
$$\hspace*{-3.2cm}
D = D_1 \cup D_2 \ \ \ 
\int\limits_{D}{f(x)dx}  = 
\int\limits_{D_1}{f(x)dx} +
\int\limits_{D_2}{f(x)dx} \ \ 
\mbox{(мера границы = 0)}
$$

\vspace*{-1.15cm}
\textbf{8)}  
\\ \\ \\ \\ \\ \\ 


\end{minipage} 

}

\end{flushleft}
\end{document}