\documentclass[12pt]{article}
\usepackage[utf8x]{inputenc}
\usepackage{cmap}
\usepackage{soul}
\usepackage{ulem}
\usepackage[english,russian]{babel}
\usepackage{enumerate}
\pagestyle{empty}
	\hoffset=0mm
	\oddsidemargin=-15mm
	\topmargin=-15mm

	\headheight=0pt
	\headsep=0pt
	\textheight=28cm
	\textwidth=16cm
	\marginparsep=0pt
	\marginparwidth=0pt
	\footskip=0pt
	\marginparpush=0pt
	\hoffset=0mm
	\voffset=0mm
	\parindent=2cm
	%\hangindent=0.5cm 
	%\hangafter=-1
	{\raggedright}
\renewcommand{\baselinestretch}{1}

\begin{document}

\textbf{Вариант 5.}
\begin{flushleft}
\framebox(550,740)[1]{

\begin{minipage}[0]{540pt}

\begin{flushright}
\textbf{{\large \textit{Глава пятая\\Избыток хлорки}}}
\end{flushright}

\vspace*{-10pt}
\hspace{3mm}\textit{Гарри и Думбльдор подошли к задней двери  \textnormal{ПРИСТАНИЩА}. Рядом, как всегда, валялись старые резиновые сапоги и ржавые котлы; из курятника неподалеку доносилось сонное квохтанье. Думбльдор постучал три раза, и в кухонном окне сейчас же промелькнул чей-то силуэт.}

	
\begin{enumerate}
\parsep=-10pt
\setcounter{enumi}{35}
\vspace*{-10pt}\item
\textbf{Кто здесь?} – сказал нервный голос, и Гарри узнал миссис Уэсли. – \textbf{Отзовитесь!}
\vspace*{-10pt}\item
\textbf{Это я, Думбльдор. Мы с Гарри.}
\end{enumerate}
\vspace*{-10pt}
\hspace{3mm}Дверь распахнулась. На пороге в старом зеленом халате стояла мама Рона, маленькая и кругленькая. 
\begin{enumerate}
\setcounter{enumi}{37}
\vspace*{-10pt}\item
\textbf{Гарри, милый! Альбус, как же вы меня напугали, вы ведь говорили, что не появитесь раньше утра!}
\vspace*{-10pt}\item
\textbf{Нам повезло, \textnormal{– сказал Думбльдор, пропуская Гарри вперед, –} Дивангард оказался сговорчивей, чем я думал. Из-за Гарри, разумеется. А, Нимфадора! Привет.}
\end{enumerate}
\vspace*{-10pt}
\hspace{3mm}\fbox{Гарри огляделся: действительно, несмотря на поздний час, миссис Уэсли была в кухне не одна}. За столом, обнимая ладонями большую кружку, сидела молодая ведьма с бледным лицом в форме сердечка и мышино-бурыми волосами. 
\begin{enumerate}
\setcounter{enumi}{39}
\vspace*{-10pt}\item
\textbf{Добрый вечер, профессор, \textnormal{– поздоровалась она. –} Салют, Гарри.}
\vspace*{-10pt}\item
\textbf{Привет, Бомс.}
\end{enumerate}
\vspace*{-10pt}
\hspace{3mm}Бомс казалась уставшей, даже больной, а ее улыбка – натянутой. Без привычных розовых, цвета жевательной резинки, волос она выглядела какой-то полинявшей.
\begin{enumerate}
\setcounter{enumi}{41}
\vspace*{-10pt}\item
\textbf{Я, пожалуй, пойду, \textnormal{– заторопилась она, встала и накинула на плечи плащ. – } $_{\mbox{\begin{tiny}
Молли\end{tiny}}}$, спасибо за чай и сочувствие.}
\vspace*{-10pt}\item
\textbf{Если ты из-за меня, то, пожалуйста, не беспокойся, \textnormal{– с обычной любезностью сказал Думбльдор. –} Я никак не могу остаться, мне нужно срочно кое-что обсудить со Скримжером.}
\vspace*{-10pt}\item
\textbf{Нет-нет, пора, \textnormal{– Бомс избегала взгляда Думбльдора. –} Всего…}
\vspace*{-10pt}\item
\textbf{Дорогая, может, поужинаешь с нами в выходные? Будут \textit{РЕМ И ШИЗОГЛАЗ…}}
\vspace*{-10pt}\item
\textbf{Нет, Молли, правда… но все равно спасибо… спокойной всем ночи.}
\end{enumerate}
\vspace*{-10pt}
\hspace{3mm}Бомс прошмыгнула мимо Гарри и Думбльдора, вышла во двор, в нескольких шагах от порога повернулась на месте и мгновенно исчезла. Гарри обратил внимание на озабоченный вид миссис Уэсли.
\begin{enumerate}
\setcounter{enumi}{46}
\vspace*{-10pt}\item
\textbf{Ну-с, Гарри, увидимся в «Хогварце», \textnormal{– произнес Думбльдор. –} Всего тебе доброго. Молли, мое почтение.}
\end{enumerate}
\vspace*{-10pt}
\hspace{3mm}Он поклонился миссис Уэсли, вышел и исчез точно там же, где и Бомс. Миссис Уэсли затворила входную дверь, взяла Гарри за плечи и повела к столу, в круг света под лампой.
\begin{enumerate}
\setcounter{enumi}{47}
\vspace*{-10pt}\item
\textbf{\textit{Совсем как Рон}, \textnormal{– вздохнула она, осмотрев Гарри с головы до ног. –} \uline{На вас будто  растяжное заклятие наложили. Рон вырос дюйма на четыре с тех пор, как мы в последний раз покупали форму. Хочешь есть, Гарри?}}

%\underline{На вас будто}  \underline{растяжное заклятие наложили. Рон вырос дюйма на четыре с тех пор, как мы в} \underline{последний раз покупали форму. Хочешь есть, Гарри?}}

\vspace*{-10pt}\item
\textbf{Да,} – ответил Гарри, только сейчас понимая, до какой степени голоден.
\vspace*{-10pt}\item
\textbf{Садись, дорогой, я сейчас что-нибудь придумаю.}
\end{enumerate}
\vspace*{-10pt}\hspace{1.5mm}
Гарри сел. К нему на колени тут же вспрыгнул мохнатый рыжий кот с приплюснутой мордой и свернулся клубком, громко мурлыча.
\begin{enumerate}
\setcounter{enumi}{50}
\vspace*{-10pt}\item
\textbf{Так Гермиона здесь?} – \sout {радостно спросил Гарри, почесывая Косолапсуса за ухом.}


%\makebox[0pt][l]{радостно спросил Гарри, почесывая Косолапсуса за ухом.}
%\makebox{-------------------------------------------------------------------------------------------------}


\vspace*{-10pt}\item
\textbf{Позавчера приехала, \textnormal{– ответила миссис Уэсли, касаясь волшебной палочкой большой железной кастрюли. Та с грохотом шлепнулась на плиту и сразу наполнилась кипящей водой. –} Сейчас все, конечно, спят, мы ждали тебя только утром. Так, давай-ка…}
\end{enumerate}
\vspace*{-10pt}
\hspace{3mm}\makebox[0pt][l]{Она еще раз стукнула по кастрюле, которая взмыла в воздух, подлетела к Гарри и перевернулась;}\makebox{\textbf{===================================================}} 
 миссис Уэсли ловко подставила тарелку и набрала в нее густого, горячего лукового супа.
\begin{enumerate}
\setcounter{enumi}{52}
\vspace*{-10pt}\item
\textbf{Хлеба, милый?}
\vspace*{-10pt}\item
\textbf{Спасибо, миссис Уэсли.}
\end{enumerate}



\end{minipage} 

}

\end{flushleft}
\end{document}
