\documentclass[12pt]{article}
\usepackage[utf8x]{inputenc}
\usepackage{cmap}
\usepackage{soul}
\usepackage{ulem}
\usepackage[english,russian]{babel}
\usepackage{enumerate}
\pagestyle{empty}
	\hoffset=0mm
	\oddsidemargin=-15mm
	\topmargin=-15mm

	\headheight=0pt
	\headsep=0pt
	\textheight=28cm
	\textwidth=20cm
	\marginparsep=0pt
	\marginparwidth=0pt
	\footskip=0pt
	\marginparpush=0pt
	\hoffset=0mm
	\voffset=0mm
	\parindent=2cm
	%\hangindent=0.5cm 
	%\hangafter=-1
	{\raggedright}
\renewcommand{\baselinestretch}{1}

\begin{document}

\begin{flushleft}
	\parindent=0.5cm
	\hangindent=-0.5cm 
	\hangafter=0
\par{\ } \\
\textbf{Вариант 5.} 

\textbf{Задача 1}
\par{Дана величина $X \sim R(-1,1)$, найти матожидание и дисперсию $Y=X^2$.\\}
\textbf{Решение}
\par{Так как $X \sim R(-1,1)$ и $Y=X^2$, то $Y \sim R(0,1)$, следовательно, 
$$M[Y] = \frac{0+1}{2} = \frac{1}{2}$$
$$D[Y] = \frac{(1-0)^2}{12} = \frac{1}{12}$$
}

\textbf{Задача 2}
\par{Решить диференциальное уравнение $y''-2y'+2y=xe^{x}\cos{x}$} \\
\textbf{Решение}
\par{Представим $y = \tilde{y} + \overline{y}$.} \\
\par{$\tilde{y}$ -- решение однородного уравнения. Решим соответствующее характеристическое уравнение:}
$$ \lambda^2 - 2\lambda + 2 = 0 \Rightarrow \lambda = 1 \pm i \Rightarrow \tilde{y} = e^{x}(c_{1}\cos{x} + c_{2}\sin{x})$$
\par{$\overline{y}$ -- Частное решение уравнения. Решим как уравнение со специальной правой частью:}
$$\overline{y} = e^{x}[(ax+b)cos{x} + (cx+d)\sin{x}]$$
$$\overline{y}' = e^x [\cos{x}(a + b + d + a x + c x) + \sin{x}(-b + c + d - a x + c x)]$$
$$\overline{y}'' = 2 e^x [\cos{x}(a + c + d + c x) + \sin{x}(a + b - c + a x)]$$
$$\overline{y}'' - 2 \overline{y}' + 2\overline{y} = xe^{x}\cos{x} \Rightarrow$$
$$\Rightarrow 
2 e^x [\cos{x}(2 b + c + 2 d + 2 c x + 2 a (1 + x)) + \sin{x}(a + 2 b - 2 c - 2 d + 2 a x - 2 c x)] = xe^{x}\cos{x}]$$
\par{Найти коэффициенты можно решив 4 линейных уравнения:}\\
$$
\left\{
\begin{array}[c]{lcl} 
2b+c+2d+2a & = & 0 \\
4c + 4a & = & 1 \\ 
a+2b-2c-2d & = & 0 \\
2a-2c & = & 0 \ \  
\end{array}\right. \Longleftrightarrow
\left\{
\begin{array}[c]{lcl} 
a & = & \frac{1}{8} \\
c & = & \frac{1}{8} \\ 
2b+2d & = & -\frac{3}{8} \\
2b-2d & = & \frac{1}{8}   
\end{array}\right. \Longleftrightarrow
\left\{
\begin{array}[c]{lcl} 
a & = & \frac{1}{8} \\
c & = & \frac{1}{8} \\ 
b & = & -\frac{1}{16} \\
d & = & \frac{1}{8}  
\end{array}\right. 
$$
\par{Ответ:}
$$
y = \frac{e^{x}}{8}[((x-2+c_1)\cos{x} + (x+1+c_2)\sin{x})], c_1, c_2 \in \Re
$$


\textbf{Задача 3}
\par{Привести к каноническому виду $ 7x^2 + 60xy +39y^2 = 0$} \\
\textbf{Решение}
\par{Найдём собственные числа и собственные векторы матрицы} 
$$\left(
\begin{array}[c]{cc} 
7 & 30\\
30 & 39
\end{array}\right) 
$$
\par{найдя корни уравнения}
$$
\left|
\begin{array}[c]{cc} 
7-\lambda & 30\\
30 & 39-\lambda
\end{array}\right| 
=0=(7-\lambda)(39-\lambda) - 900 = 273-46\lambda+\lambda^{2} - 900 = \lambda^2 - 46\lambda - 627
$$
\par{Собственные числа $\lambda_{1} = 57, \lambda_{2} = -11$}
\par{Собственный вектор $v_{1}$:}
$$\left(
\begin{array}[c]{cc} 
-50 & 30\\
30 & -18
\end{array}\right) 
v_{1} = 0 \Rightarrow v_{1} = 
\left(\begin{array}[c]{c} 
-\frac{5}{3} \\
1
\end{array}\right),  v_{1}^{normalized}  = \frac{1}{\sqrt{34}}
\left(\begin{array}[c]{c} -5\\3\end{array}\right)
$$
\par{Собственный вектор $v_{2}$:}
$$\left(
\begin{array}[c]{cc} 
18 & 30\\
30 & 50
\end{array}\right) 
v_{2} = 0 \Rightarrow v_{2} = 
\left(\begin{array}[c]{c} 
\frac{3}{5} \\1\end{array}\right),
v_{2}^{normalized}  = \frac{1}{\sqrt{34}}
\left(\begin{array}[c]{c} 3\\5\end{array}\right)
$$
\par{Преобразование координат:}
$$\left\{
\begin{array}[c]{rcl} 
x & = & \frac{1}{\sqrt{34}}(-5x'+3y')\\
y & = & \frac{1}{\sqrt{34}}(3x'+5y')
\end{array}\right.
$$
$$
\begin{array}[c]{cc} 
7x^2 + 60xy +39y^2 = 0 = \frac{1}{34}(7(-5x'+3y')^{2} + 60(-5x'+3y')(3x'+5y') + 39(3x'+5y')^{2}) = \\
=-11*(x')^{2}+57*(y')^{2}
\end{array}
$$
\par{Ответ: уравнение пересекающихся прямых}
$$
\frac{x^{2}}{57} - \frac{y^{2}}{11} = 0
$$\\
\textbf{Задача 4}
\par{Используя квадратурную формулу Гаусса с 5 узлами найти определённый интеграл } 
$$
\int\limits_{1}^{6}{\frac{\sqrt{x^2-4}}{x}}dx
$$\\
\textbf{Решение}
\par{Преобразуем интервал интегрирования от (1,6) к (-1,1), введя замену $y=\frac{2x-7}{5}$}: 
$$I = 
\int\limits_{1}^{6}{\frac{\sqrt{x^2-4}}{x}}dx=
\frac{5}{2}\int\limits_{-1}^{1}{ \frac{\sqrt{\left(\frac{5x+7}{2}\right)^{2}-4}}{\frac{5x+7}{2}}}dx = 
= \frac{5}{2}\int\limits_{-1}^{1}{\frac{\sqrt{(5x+7)^{2}-16}}{5x+7}}dx = \frac{5}{2}\int\limits_{-1}^{1}{f(x)dx}
$$
\par{Используя квадратурную формулу Гаусса с 5 узлами,}
$$
\begin{array}[c]{c}
I \approx \frac{5}{2}\sum_{i=1}^{5}{c_{i}f(x_{i})}= 
 \frac{2}{5}(0.236927f(-0.906180) + 0,478629f(-0.538469) + \\
+0.568889f(0) +0,478629f(0.538469) + 0.236927f(0.906180) = \\
=3.256695045+0.7549378995i
\end{array}
$$
\par{Ответ:}
$$
\int\limits_{1}^{6}{\frac{\sqrt{x^2-4}}{x}}dx \approx 3.256695045+0.7549378995i
$$


\end{flushleft}
\end{document}